% Options for packages loaded elsewhere
\PassOptionsToPackage{unicode}{hyperref}
\PassOptionsToPackage{hyphens}{url}
%
\documentclass[
]{article}
\usepackage{lmodern}
\usepackage{amssymb,amsmath}
\usepackage{ifxetex,ifluatex}
\ifnum 0\ifxetex 1\fi\ifluatex 1\fi=0 % if pdftex
  \usepackage[T1]{fontenc}
  \usepackage[utf8]{inputenc}
  \usepackage{textcomp} % provide euro and other symbols
\else % if luatex or xetex
  \usepackage{unicode-math}
  \defaultfontfeatures{Scale=MatchLowercase}
  \defaultfontfeatures[\rmfamily]{Ligatures=TeX,Scale=1}
\fi
% Use upquote if available, for straight quotes in verbatim environments
\IfFileExists{upquote.sty}{\usepackage{upquote}}{}
\IfFileExists{microtype.sty}{% use microtype if available
  \usepackage[]{microtype}
  \UseMicrotypeSet[protrusion]{basicmath} % disable protrusion for tt fonts
}{}
\makeatletter
\@ifundefined{KOMAClassName}{% if non-KOMA class
  \IfFileExists{parskip.sty}{%
    \usepackage{parskip}
  }{% else
    \setlength{\parindent}{0pt}
    \setlength{\parskip}{6pt plus 2pt minus 1pt}}
}{% if KOMA class
  \KOMAoptions{parskip=half}}
\makeatother
\usepackage{xcolor}
\IfFileExists{xurl.sty}{\usepackage{xurl}}{} % add URL line breaks if available
\IfFileExists{bookmark.sty}{\usepackage{bookmark}}{\usepackage{hyperref}}
\hypersetup{
  pdftitle={Simulation},
  pdfauthor={NithaDuff},
  hidelinks,
  pdfcreator={LaTeX via pandoc}}
\urlstyle{same} % disable monospaced font for URLs
\usepackage[margin=1in]{geometry}
\usepackage{color}
\usepackage{fancyvrb}
\newcommand{\VerbBar}{|}
\newcommand{\VERB}{\Verb[commandchars=\\\{\}]}
\DefineVerbatimEnvironment{Highlighting}{Verbatim}{commandchars=\\\{\}}
% Add ',fontsize=\small' for more characters per line
\usepackage{framed}
\definecolor{shadecolor}{RGB}{248,248,248}
\newenvironment{Shaded}{\begin{snugshade}}{\end{snugshade}}
\newcommand{\AlertTok}[1]{\textcolor[rgb]{0.94,0.16,0.16}{#1}}
\newcommand{\AnnotationTok}[1]{\textcolor[rgb]{0.56,0.35,0.01}{\textbf{\textit{#1}}}}
\newcommand{\AttributeTok}[1]{\textcolor[rgb]{0.77,0.63,0.00}{#1}}
\newcommand{\BaseNTok}[1]{\textcolor[rgb]{0.00,0.00,0.81}{#1}}
\newcommand{\BuiltInTok}[1]{#1}
\newcommand{\CharTok}[1]{\textcolor[rgb]{0.31,0.60,0.02}{#1}}
\newcommand{\CommentTok}[1]{\textcolor[rgb]{0.56,0.35,0.01}{\textit{#1}}}
\newcommand{\CommentVarTok}[1]{\textcolor[rgb]{0.56,0.35,0.01}{\textbf{\textit{#1}}}}
\newcommand{\ConstantTok}[1]{\textcolor[rgb]{0.00,0.00,0.00}{#1}}
\newcommand{\ControlFlowTok}[1]{\textcolor[rgb]{0.13,0.29,0.53}{\textbf{#1}}}
\newcommand{\DataTypeTok}[1]{\textcolor[rgb]{0.13,0.29,0.53}{#1}}
\newcommand{\DecValTok}[1]{\textcolor[rgb]{0.00,0.00,0.81}{#1}}
\newcommand{\DocumentationTok}[1]{\textcolor[rgb]{0.56,0.35,0.01}{\textbf{\textit{#1}}}}
\newcommand{\ErrorTok}[1]{\textcolor[rgb]{0.64,0.00,0.00}{\textbf{#1}}}
\newcommand{\ExtensionTok}[1]{#1}
\newcommand{\FloatTok}[1]{\textcolor[rgb]{0.00,0.00,0.81}{#1}}
\newcommand{\FunctionTok}[1]{\textcolor[rgb]{0.00,0.00,0.00}{#1}}
\newcommand{\ImportTok}[1]{#1}
\newcommand{\InformationTok}[1]{\textcolor[rgb]{0.56,0.35,0.01}{\textbf{\textit{#1}}}}
\newcommand{\KeywordTok}[1]{\textcolor[rgb]{0.13,0.29,0.53}{\textbf{#1}}}
\newcommand{\NormalTok}[1]{#1}
\newcommand{\OperatorTok}[1]{\textcolor[rgb]{0.81,0.36,0.00}{\textbf{#1}}}
\newcommand{\OtherTok}[1]{\textcolor[rgb]{0.56,0.35,0.01}{#1}}
\newcommand{\PreprocessorTok}[1]{\textcolor[rgb]{0.56,0.35,0.01}{\textit{#1}}}
\newcommand{\RegionMarkerTok}[1]{#1}
\newcommand{\SpecialCharTok}[1]{\textcolor[rgb]{0.00,0.00,0.00}{#1}}
\newcommand{\SpecialStringTok}[1]{\textcolor[rgb]{0.31,0.60,0.02}{#1}}
\newcommand{\StringTok}[1]{\textcolor[rgb]{0.31,0.60,0.02}{#1}}
\newcommand{\VariableTok}[1]{\textcolor[rgb]{0.00,0.00,0.00}{#1}}
\newcommand{\VerbatimStringTok}[1]{\textcolor[rgb]{0.31,0.60,0.02}{#1}}
\newcommand{\WarningTok}[1]{\textcolor[rgb]{0.56,0.35,0.01}{\textbf{\textit{#1}}}}
\usepackage{graphicx,grffile}
\makeatletter
\def\maxwidth{\ifdim\Gin@nat@width>\linewidth\linewidth\else\Gin@nat@width\fi}
\def\maxheight{\ifdim\Gin@nat@height>\textheight\textheight\else\Gin@nat@height\fi}
\makeatother
% Scale images if necessary, so that they will not overflow the page
% margins by default, and it is still possible to overwrite the defaults
% using explicit options in \includegraphics[width, height, ...]{}
\setkeys{Gin}{width=\maxwidth,height=\maxheight,keepaspectratio}
% Set default figure placement to htbp
\makeatletter
\def\fps@figure{htbp}
\makeatother
\setlength{\emergencystretch}{3em} % prevent overfull lines
\providecommand{\tightlist}{%
  \setlength{\itemsep}{0pt}\setlength{\parskip}{0pt}}
\setcounter{secnumdepth}{-\maxdimen} % remove section numbering

\title{Simulation}
\author{NithaDuff}
\date{7/23/2020}

\begin{document}
\maketitle

\begin{Shaded}
\begin{Highlighting}[]
\KeywordTok{library}\NormalTok{(dplyr)}
\end{Highlighting}
\end{Shaded}

\begin{verbatim}
## 
## Attaching package: 'dplyr'
\end{verbatim}

\begin{verbatim}
## The following objects are masked from 'package:stats':
## 
##     filter, lag
\end{verbatim}

\begin{verbatim}
## The following objects are masked from 'package:base':
## 
##     intersect, setdiff, setequal, union
\end{verbatim}

\begin{Shaded}
\begin{Highlighting}[]
\KeywordTok{library}\NormalTok{(ggplot2)}
\end{Highlighting}
\end{Shaded}

\begin{verbatim}
## Warning: package 'ggplot2' was built under R version 4.0.2
\end{verbatim}

\hypertarget{toothgrowth-data-analysis}{%
\subsection{ToothGrowth data analysis}\label{toothgrowth-data-analysis}}

Here we are analysing the data reported for the ToothGrowth of a sample
of 60 guinea pigs with varying dosage and 2 alternative supplements. The
data looks like the following table.

\begin{Shaded}
\begin{Highlighting}[]
\KeywordTok{data}\NormalTok{(}\StringTok{"ToothGrowth"}\NormalTok{)}
\KeywordTok{head}\NormalTok{(ToothGrowth)}
\end{Highlighting}
\end{Shaded}

\begin{verbatim}
##    len supp dose
## 1  4.2   VC  0.5
## 2 11.5   VC  0.5
## 3  7.3   VC  0.5
## 4  5.8   VC  0.5
## 5  6.4   VC  0.5
## 6 10.0   VC  0.5
\end{verbatim}

\begin{Shaded}
\begin{Highlighting}[]
\KeywordTok{summary}\NormalTok{(ToothGrowth)}
\end{Highlighting}
\end{Shaded}

\begin{verbatim}
##       len        supp         dose      
##  Min.   : 4.20   OJ:30   Min.   :0.500  
##  1st Qu.:13.07   VC:30   1st Qu.:0.500  
##  Median :19.25           Median :1.000  
##  Mean   :18.81           Mean   :1.167  
##  3rd Qu.:25.27           3rd Qu.:2.000  
##  Max.   :33.90           Max.   :2.000
\end{verbatim}

As seen above the data frame has 3 columns each containing the length(of
tooth), supplement group, dosage given. There are 2 supplements and 3
variant dosages of each supplement. This is equally divided between 60
guinea pigs.

\begin{Shaded}
\begin{Highlighting}[]
\KeywordTok{t.test}\NormalTok{(len }\OperatorTok{~}\StringTok{ }\KeywordTok{I}\NormalTok{(}\KeywordTok{relevel}\NormalTok{(supp,}\DecValTok{2}\NormalTok{)), }\DataTypeTok{paired =} \OtherTok{FALSE}\NormalTok{, }\DataTypeTok{data =}\NormalTok{ ToothGrowth)}
\end{Highlighting}
\end{Shaded}

\begin{verbatim}
## 
##  Welch Two Sample t-test
## 
## data:  len by I(relevel(supp, 2))
## t = -1.9153, df = 55.309, p-value = 0.06063
## alternative hypothesis: true difference in means is not equal to 0
## 95 percent confidence interval:
##  -7.5710156  0.1710156
## sample estimates:
## mean in group VC mean in group OJ 
##         16.96333         20.66333
\end{verbatim}

from the above data, we can analyse the group means, the confidence
interval of the entire data and so on.

\hypertarget{toothgrowth-data-visualisation}{%
\subsection{ToothGrowth data
Visualisation}\label{toothgrowth-data-visualisation}}

\begin{Shaded}
\begin{Highlighting}[]
\NormalTok{tooth_mns <-}\StringTok{ }\NormalTok{ToothGrowth }\OperatorTok\StringTok{ }\KeywordTok{group_by}\NormalTok{(supp,dose)  }\OperatorTok\StringTok{ }\KeywordTok{summarise}\NormalTok{(}\DataTypeTok{mean =} \KeywordTok{mean}\NormalTok{(len)) }\OperatorTok\StringTok{ }\KeywordTok{print}\NormalTok{()}
\end{Highlighting}
\end{Shaded}

\begin{verbatim}
## `summarise()` regrouping output by 'supp' (override with `.groups` argument)
\end{verbatim}

\begin{verbatim}
## # A tibble: 6 x 3
## # Groups:   supp [2]
##   supp   dose  mean
##   <fct> <dbl> <dbl>
## 1 OJ      0.5 13.2 
## 2 OJ      1   22.7 
## 3 OJ      2   26.1 
## 4 VC      0.5  7.98
## 5 VC      1   16.8 
## 6 VC      2   26.1
\end{verbatim}

\begin{Shaded}
\begin{Highlighting}[]
\NormalTok{g <-}\StringTok{ }\KeywordTok{ggplot}\NormalTok{(ToothGrowth, }\KeywordTok{aes}\NormalTok{(}\DataTypeTok{x =}\NormalTok{ dose, }\DataTypeTok{y =}\NormalTok{ len, }\DataTypeTok{colour =}\NormalTok{ dose, }\DataTypeTok{group =}\NormalTok{ supp))}
\NormalTok{g <-}\StringTok{ }\NormalTok{g }\OperatorTok{+}\StringTok{ }\KeywordTok{geom_point}\NormalTok{()}
\NormalTok{g <-}\StringTok{ }\NormalTok{g }\OperatorTok{+}\StringTok{ }\KeywordTok{geom_line}\NormalTok{(tooth_mns, }\DataTypeTok{mapping =} \KeywordTok{aes}\NormalTok{(}\DataTypeTok{x =}\NormalTok{ dose,}\DataTypeTok{y =}\NormalTok{ mean),}\DataTypeTok{linetype =} \DecValTok{2}\NormalTok{, }\DataTypeTok{size =} \DecValTok{1}\NormalTok{, }\DataTypeTok{color =} \StringTok{"red"}\NormalTok{) }
\NormalTok{g <-}\StringTok{ }\NormalTok{g }\OperatorTok{+}\StringTok{ }\KeywordTok{facet_grid}\NormalTok{(. }\OperatorTok{~}\StringTok{ }\NormalTok{supp) }\OperatorTok{+}\StringTok{ }\KeywordTok{labs}\NormalTok{(}\DataTypeTok{x =} \StringTok{"Dosage"}\NormalTok{, }\DataTypeTok{y =} \StringTok{"Length"}\NormalTok{ , }\DataTypeTok{title =} \StringTok{"Comparison in supplement vs dosage"}\NormalTok{)}
\KeywordTok{print}\NormalTok{(g)}
\end{Highlighting}
\end{Shaded}

\includegraphics{BasicDiff_files/figure-latex/tg-1.pdf}

The plot here shows the comparison between the 2 supplement types with
respect to the level of dosage. From the plot we can infer that in
either case of supplement the dosage with value 2.0 has a higher mean.
Though the difference in means between 1.0 and 2.0 in case of supplement
OJ is very less. Hence the increase in dosage has not made much impact.
Whereas in case of supplement VC, the dosage levels show a significant
improvement in length.

\begin{Shaded}
\begin{Highlighting}[]
\NormalTok{g <-}\StringTok{ }\KeywordTok{ggplot}\NormalTok{(ToothGrowth, }\KeywordTok{aes}\NormalTok{(}\DataTypeTok{x =} \KeywordTok{factor}\NormalTok{(dose), }\DataTypeTok{y =}\NormalTok{ len, }\DataTypeTok{fill =} \KeywordTok{factor}\NormalTok{(dose)))}
\NormalTok{g <-}\StringTok{ }\NormalTok{g }\OperatorTok{+}\StringTok{ }\KeywordTok{geom_violin}\NormalTok{(}\DataTypeTok{col =} \StringTok{"black"}\NormalTok{, }\DataTypeTok{size =} \DecValTok{2}\NormalTok{)}
\NormalTok{g <-}\StringTok{ }\NormalTok{g }\OperatorTok{+}\StringTok{ }\KeywordTok{facet_grid}\NormalTok{(. }\OperatorTok{~}\StringTok{ }\NormalTok{supp) }\OperatorTok{+}\StringTok{ }\KeywordTok{labs}\NormalTok{(}\DataTypeTok{x =} \StringTok{"Dosage"}\NormalTok{, }\DataTypeTok{y =} \StringTok{"Length"}\NormalTok{ , }\DataTypeTok{title =} \StringTok{"Comparison in supplement vs dosage - density"}\NormalTok{)}
\KeywordTok{print}\NormalTok{(g)}
\end{Highlighting}
\end{Shaded}

\includegraphics{BasicDiff_files/figure-latex/net_violin-1.pdf}

The violin plots shows that the net effect dosage of VC with value 2.0
has a higher limit. Yet he results of supplement OJ has a comparatively
steady outcome. with varying dosages.

\end{document}
